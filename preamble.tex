\newtheoremstyle{myplain}{10pt}{10pt}{\itshape}{}{\scshape}{.}{.5em}{}
\newtheoremstyle{mydefinition} {10pt}{10pt}{\itshape}{}{\scshape}{.}{.5em}{}
\newtheoremstyle{myremark}     {10pt}{10pt}{}{}{\scshape}{.}{.5em}{}
\newcounter{lecture}
%\theoremstyle{myplain}

\newtheorem{theorem}{Theorem}[section]
\newtheorem{lemma}[theorem]{Lemma}
\newtheorem{proposition} [theorem]{Proposition}
\newtheorem{corollary}[theorem]{Corollary}
\newtheorem{claim}[theorem]{Claim}
\newtheorem{fact} [theorem]{Fact}
%\theoremstyle{mydefinition}
\newtheorem{definition} [theorem]{Definition}
\newtheorem{example}[theorem]{Example}
\newtheorem{assumption}[theorem]{Assumption}
\newtheorem{openproblem}[theorem]{Open Problem}
\newtheorem{problem} [theorem]{Problem}
%\theoremstyle{myremark}
\newtheorem{remark} [theorem]{Remark}
\newtheorem{conjecture} [theorem]{Conjecture}
\newtheorem{observation}[theorem]{Observation}
\newtheorem{property}{Property}[section]
\newtheorem{recurrence relation}{Recurrence Relation}[section]
%\newtheorem{exercise}{Exercise}

\numberwithin{equation} {lecture}
\numberwithin{figure}{lecture}
\numberwithin{table}{lecture}
\newcommand{\cupdot}{\mathbin{\mathaccent\cdot\cup}}
\newcommand{\bigsum}{\mathlarger{\mathlarger{\sum}}}
\newcommand{\bigger}[1]{\mathlarger{\mathlarger{#1}}}
\renewcommand{\bar}[1]{\overline{\vphantom{1^a}#1}}
%\renewcommand{\fnum@figure}{\textsc{Figure~\thefigure}}
%\renewcommand{\fnum@table}{\textsc{Table~\thetable}}

\theoremstyle{plain}

\newenvironment{proof-sketch}{\noindent{\bf Sketch of Proof}\hspace*{1em}}{\qed\bigskip}
\newenvironment{proof-idea}{\noindent{\bf Proof Idea}\hspace*{1em}}{\qed\bigskip}
\newenvironment{proof-of-lemma}[1]{\noindent{\bf Proof of Lemma #1}\hspace*{1em}}{\qed\bigskip}
\newenvironment{proof-attempt}{\noindent{\bf Proof Attempt}\hspace*{1em}}{\qed\bigskip}
\newenvironment{proofof}[1]{\noindent{\bf Proof}
	of #1:\hspace*{1em}}{\qed\bigskip}

\renewcommand{\qedsymbol}{\leavevmode
	\hbox to.77778em{%
		\hfil\vrule
		\vbox to.875em{\hrule width.35em\vfil\hrule}%
		\vrule\hfil}}

%%%%%%%%%%%%%%%%%%%%%%%%%%%%%%%%%%%%%%%%%%%%%%%%%%%
% Useful Macros
%%%%%%%%%%%%%%%%%%%%%%%%%%%%%%%%%%%%%%%%%%%%%%%%%%%
\newcommand{\Exp}[1]{\mathify{\mbox{Exp}\left[#1\right]}}
\newcommand{\bigO}O
%\newcommand{\set}[1]{\mathify{\left\{ #1 \right\}}}
\def\half{\frac{1}{2}}
\newcommand{\V}[1]{\mathsf{Var}[#1]}
\def\implies{\Rightarrow}
\def\prob#1#2{{\mathop{{\rm Prob}}_{#1}}\left[#2 \right]}
\def\var#1#2{{\mathop{{\rm Var}}_{#1}}[#2]}
\def\expec#1#2{{\mathop{{\rm E}}_{#1}}[#2]}
\def\sizeof#1{\left| #1\right|}
\def\setof#1{\left\{ #1\right\}  }
\newcommand\norm[1]{{\left\lVert#1\right\rVert}_2}
\newcommand{\F}{{\mathbb{F}}}
\newcommand{\Z}{{\mathbb{Z}}}
\newcommand{\supp}{{\mathsf{supp}}}
%\newcommand{\qed}{\rule{7pt}{7pt}}
\newcommand*\circled[1]{~\tikz[baseline=(char.base)]{
		\node[shape=circle,draw,inner sep=1.5pt] (char) {\tiny #1};}~}

\newcommand{\FOR}{{\bf for}}
\newcommand{\TO}{{\bf to}}
\newcommand{\DO}{{\bf do}}
\newcommand{\WHILE}{{\bf while}}
\newcommand{\AND}{{\bf and}}
\newcommand{\IF}{{\bf if}}
\newcommand{\THEN}{{\bf then}}
\newcommand{\ELSE}{{\bf else}}
\newcommand{\N}{\mathbb{N}}

%newly defined commands by Narasimha Sai
\newcommand{\mulnom}[2]{$\left(\binom{#1}{#2}\right)$}
\newcommand{\vecx}{\mathbf{x}}
%ends here

% \renewcommand{\thefigure}{\thesection.\arabic{figure}}
% \renewcommand{\thetable}{\thesection.\arabic{table}}
% \renewcommand{\theequation}{\thesection.\arabic{equation}}

% Calligraphic letters
\newcommand{\calA}{{\cal A}}
\newcommand{\calB}{{\cal B}}
\newcommand{\calC}{{\cal C}}
\newcommand{\calD}{{\cal D}}
\newcommand{\calE}{{\cal E}}
\newcommand{\calF}{{\cal F}}
\newcommand{\calG}{{\cal G}}
\newcommand{\calH}{{\cal H}}
\newcommand{\calI}{{\cal I}}
\newcommand{\calJ}{{\cal J}}
\newcommand{\calK}{{\cal K}}
\newcommand{\calL}{{\cal L}}
\newcommand{\calM}{{\cal M}}
\newcommand{\calN}{{\cal N}}
\newcommand{\calO}{{\cal O}}
\newcommand{\calP}{{\cal P}}
\newcommand{\calQ}{{\cal Q}}
\newcommand{\calR}{{\cal R}}
\newcommand{\calS}{{\cal S}}
\newcommand{\calT}{{\cal T}}
\newcommand{\calU}{{\cal U}}
\newcommand{\calV}{{\cal V}}
\newcommand{\calW}{{\cal W}}
\newcommand{\calX}{{\cal X}}
\newcommand{\calY}{{\cal Y}}
\newcommand{\calZ}{{\cal Z}}


\setcounter{tocdepth}{3}
\setcounter{secnumdepth}{2}
\sloppy

\newcommand{\AsymCloud}[3]{
	\begin{scope}[shift={#1},scale=#3]
		\draw (-1.6,-0.7) .. controls (-2.3,-1.1)
		and (-2.7,0.3) .. (-1.7,0.3)coordinate(asy1) .. controls (-1.6,0.7)
		and (-1.2,0.9) .. (-0.8,0.7) .. controls (-0.5,1.5)
		and (0.6,1.3) .. (0.7,0.5) .. controls (1.5,0.4)
		and (1.2,-1) .. (0.4,-0.6)coordinate(asy2) .. controls (0.2,-1)
		and (-0.2,-1) .. (-0.5,-0.7) .. controls (-0.9,-1)
		and (-1.3,-1) .. cycle;
		\node at ($(asy1)!0.5!(asy2)$) {#2};
	\end{scope}
}



\AtBeginDocument{\renewcommand\contentsname{Table of Contents}}

\newcommand{\listofscribes}{List of Scribes}
\newcommand{\listofinstr}{List of Instructors}
\newlistof{scribe}{scr}{\listofscribes}
\newlistof{instructors}{instr}{\listofinstr}

\newcommand{\Lecture}[7]{
	\newpage
	\setcounter{chapter}{#3}
	\setcounter{lecture}{#3}
	
	% To reset numbering in a new lecture
	\setcounter{theorem}{0}  
	% To reset numbering within a section
	\setcounter{section}{0}  
	
	% For table of contents
	\cleardoublepage	% for two sided document with odd no. of TOC pages
	\phantomsection	% for fixing hyperref link
	\addcontentsline{toc}{chapter}{Lecture \protect\numberline{#3}(#6) {#4}}
	
	% For list of scribes
	\cleardoublepage	% for two sided document with odd no. of TOC pages
	\phantomsection		% for fixing hyperref link
	
	% Adding the correctly colored line to scribe list.
	
	\ifthenelse{\equal{#6}{$\alpha$}}{\addcontentsline{scr}{scribe}{\protect {\sf Lecture } \numberline{\sf \thelecture} {\color{alpha}{#5} - (#6) \tiny{TA : #7}}}}
	{
		\ifthenelse{\equal{#6}{$\beta$}}{\addcontentsline{scr}{scribe}{\protect {\sf Lecture } \numberline{\sf \thelecture} {\color{beta}{#5} - (#6) \tiny{TA : #7}}}}
		{
			\ifthenelse{\equal{#6}{$\gamma$}}{\addcontentsline{scr}{scribe}{\protect {\sf Lecture } \numberline{\sf \thelecture} {\color{gamma}{#5} - (#6) \tiny{TA : #7}}}}
			{
				\ifthenelse{\equal{#6}{$\delta$}}{\addcontentsline{scr}{scribe}{\protect {\sf Lecture } \numberline{\sf \thelecture} {\color{delta}{#5} - (#6) \tiny{TA : #7}}}}
				{
					\addcontentsline{scr}{scribe}{\protect {\sf Lecture } \numberline{\sf \thelecture} {{#5} - (Incorrect Labelling) \tiny{TA: #7}}}
	}}}}
	
	%\addcontentsline{scr}{scribe}{\protect {\sf Lecture } \numberline{\sf \thelecture} {\em #5} (#6)}
	
	% For list of instructors
	\cleardoublepage	% for two sided document with odd no. of TOC pages
	\phantomsection		% for fixing hyperref link
	\addcontentsline{instr}{instructors}{\protect {\sf Lecture } \numberline{\sf \thelecture} {\em #1}}
	\noindent
	\parbox{9cm}{
		%   Indian Institute of Technology Madras\\
		%	CS5130 - Mathematical Tools for Theoretical Computer Science\\
		%	\rule{0mm}{6mm}%
		{\it \bf Instructor :} #1 \\
		{\it \bf Scribe :} #5 ({\it TA:} #7) \\
		{\it \bf Date :}     #2 \\
		{\it \bf Status :} #6
	}
	\hfill
	\begin{tabular}{c@{}}
		{\bf\Large Lecture}\\
		\rule{0mm}{17mm}\scalebox{6.6}{\bf\thelecture}
	\end{tabular}
	
	\vspace{1.5cm}
	\begin{center}
		{\Large \bf #4}
	\end{center}
	\vspace{1cm}
	%\thispagestyle{empty}
}

%
%\newpage
%\listofscribe          % For automatic scribe list generation.
%
%\newpage
%\listofinstr
%\newpage
%\tableofcontents
%
%\newpage 
%\pagenumbering{arabic}  % Arabic page numbering for lectures.
%%\part{Introduction, Motivation and the Language}
%\newpage \setcounter{page}{1} 
