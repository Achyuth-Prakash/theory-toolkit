\Lecture{Jayalal Sarma}{Oct 21, 2020}{17}{Generating Functions(continued)}{Pragnya}{$\alpha$}{JS}

\section{Introduction}
In this section we'll see some examples of ordinary generating functions and get introduced to exponential generating functions.

\subsection{Example 3 : }
To show two combinatorial qualities are equal it sufficies to show that they have same generating functions. Consider the following,
$$ B_n(m) = \{(x_1, x_2, \dots x_n) ~|~ \forall i ~ x_i \in \Z , \sum |x_i| \leq m \}$$
let $b_{n,m} = |B_n(m)|$. Let's see properties of  $b_{n,m}$ :
\begin{enumerate}
    \item $b_{n, m} = \sum_{k=0}^n {n \choose k}{m \choose k} 2^k$.
    \item $b_{n, m} = b_{m, n}$. This can also be proved using bijection.
    \item $b_{n, m} = d_{m, n}$. 
\end{enumerate}
We'll prove property $3$ by showing they have same generating functions.
\begin{align*}
    B_{x,y} &= \sum_{n,m \geq 0} b_{n, m} x^n y^m\\
    &= \sum_{n,m \geq 0} (\sum_{k=0}^n {n \choose k}{m \choose k} 2^k) x^n y^m \\
    &= \sum_{n,m,k \geq 0}{n \choose k}{m \choose k} 2^k x^n y^m \\
    &= \sum_{k \geq 0} 2^k \sum_{n,m \geq 0} {n \choose k}{m \choose k} x^n y^m \\
    &= \sum_{k \geq 0} 2^k (\sum_{n \geq 0} {n \choose k} x^n)(\sum_{m \geq 0} {m \choose k} y^m) \\
B_{x, y}&= \sum_{k \geq 0} 2^k (x^k \sum_{n \geq 0} {n \choose k} x^{n-k})(y^k \sum_{m \geq 0} {m \choose k} y^{m-k})
\end{align*}
Consider $\frac{1}{(1-x)^{k+1}}$ : 
$$\frac{1}{(1-x)^{k+1}} = \frac{1}{(1-x)}.\frac{1}{(1-x)}. \dots \frac{1}{(1-x)} ~(k+1 ~times)$$
Coefficient of $x^{n-k}$ in $\frac{1}{(1-x)^{k+1}}$ is equivalent to no.of solutions of $a_1 + a_2 + \dots + a_{k+1} = n-k$ which is $ = {(n-k) + (k+1) -1 \choose n-k} = {n \choose k}$.
Hence
$$ \sum_{n \geq 0} {n \choose k} x^{n-k} = \frac{1}{(1-x)^{k+1}} $$
Similarly 
$$ \sum_{m \geq 0} {m \choose k} y^{m-k} = \frac{1}{(1-y)^{k+1}} $$
Substituting them in the above derived $B_{x, y}$ - 
\begin{align*}
    B_{x, y} &= \sum_{k \geq 0} 2^k x^k y^k \frac{1}{(1-x)^{k+1}} \frac{1}{(1-y)^{k+1}} \\
    &= \sum_{k \geq 0} (2xy)^k \frac{1}{(1-x)^{k+1}} \frac{1}{(1-y)^{k+1}} \\
    &= \frac{1}{(1-x)(1-y)} \sum_{k \geq 0} \frac{(2xy)^k}{(1-x)^k(1-y)^k} \\
    &= \frac{1}{(1-x)(1-y)} \sum_{k \geq 0} (\frac{(2xy)}{(1-x)(1-y)}) ^k \\
    &= \frac{1}{(1-x)(1-y)} \frac{1}{1- \frac{2xy}{(1-x)(1-y)}} \\
    &= \frac{1}{(1-x)(1-y) - 2xy} \\
    B(x, y) &=  \frac{1}{1 - x - y - xy} = D(x, y)
\end{align*}
Since $b_{n, m}$ and $d_{n, m}$ have same generating functions, $b_{n, m} = d_{n, m}$. Hence $b_{n, m}$ also satisfies recurrence relation of $d_{n, m}$ - 
$$ b_{n, m} = b_{n-1, m} + b_{n, m-1} + b_{n-1, m-1}$$


\subsection{Example 4 : Stirling number of second kind} 
As discussed in previous lectures, number of ways to partition set $\{1, 2, 3 \dots n \}$ into $k$ non-empty parts is called stirling number of second kind. Let's represent by $S_{n,k}$. It's recurrence relation is given by -  
$$S_{n, k} = S_{n-1, k-1} + k S_{n-1, k} $$
LHS : number of ways to partition set $\{1, 2, 3 \dots n \}$ into $k$ non-empty parts $ = S_{n, k}$ \\
RHS : \begin{parts}
\item If element $1$ occurs in a singleton set. No.of ways to partition remaining $n-1$ elements to $k-1$ sets $= S_{n-1, k-1}$.
\item If element $1$ doesn't occur in a singleton set. Then we can partition remaining $n-1$ elements to $k$ sets and add element $1$ to one of these $k$ sets $=  k S_{n-1, k}$
\end{parts}
We can also see that $S_{0,0} = 1 ,~ S_{n, 0} = 0 ,~ S_{0, k} = 0$.
%$S_{x, y} = {\sum_{n, k \geq 0} ^ {\infty} S_{n,k} x^n y^k}$ \\
%\begin{equation}
\begin{align*}
 S(x, y) &= {\sum_{n, k \geq 0} ^ {\infty} S_{n,k} x^n y^k} \\
&= S_{0,0}~x^0y^0 + {\sum_{n = 0, k \geq 1}^ {\infty} S_{0,k}~x^0y^k} + {\sum_{n \geq 1, k = 0}^ {\infty} S_{n,0}~x^n y^0} + {\sum_{n \geq 1, k \geq 1}^ {\infty} S_{n,k}~x^n y^k}  \\
 &= 1 + \sum_{n \geq 1 , k \geq 1} ^ {\infty} S_{n,k} ~ x^n y^k  \\
 &= 1 + \sum_{n \geq 1 , k \geq 1} S_{n-1, k-1}~ x^n y^k + \sum_{n \geq 1, k \geq 1} k S_{n-1, k}~ x^n y^k \\
 &= 1 + xy \sum_{n \geq 1, k \geq 1} S_{n-1, k-1} x^{n-1} y^{k-1} + x\sum_{n \geq 1, k \geq 1} k S_{n-1, k}~ x^{n-1} y^k \\
 &= 1 + xy~ S(x, y) + x \sum_{n \geq 0, k \geq 1} k S_{n, k}~ x^{n} y^k \\
 &= 1 + xy~ S(x, y) + \frac{\partial}{\partial y} S(x,y)
\end{align*}
Note : $\frac{\partial}{\partial y} S(x, y) = \sum_{n \geq 0, k \geq 1} k S_{n,k}~x^n y^{k-1}$ 


Consider $y^k$ coefficients on both sides : 
\begin{equation}
  \begin{split}
    LHS &= \sum_{n \geq 0} S_{n, k} x^n \\
    RHS &= x\sum_{n \geq 0} S_{n, k-1} x^n + xk \sum_{n \geq 0} S_{n, k}x^n
\end{split}  
\end{equation}
Equating LHS and RHS :
\begin{align*}
\sum_{n \geq 0} S_{n, k}~ x^n &= x\sum_{n \geq 0} S_{n, k-1} ~x^n + xk \sum_{n \geq 0} S_{n, k}~x^n \\
\sum_{n \geq 0} S_{n, k} ~x^n &= \frac{x}{1-xk}  \sum_{n \geq 0} S_{n, k-1}~ x^n   \\
&= \frac{x}{1-xk} \frac{x}{1-x(k-1)} \sum_{n \geq 0} S_{n, k-2}~ x^n \\
&= \frac{x}{1-xk} \frac{x}{1-x(k-1)} \dots \frac{x}{1-x(k-(k-1))}
\sum_{n \geq 0} S_{n, 0}~ x^n \\
&= \frac{x^k}{(1-x)(1-2x)\dots (1-kx)} \times 1 ~( Note : S_{0,0} = 1, S_{n, 0} = 0) \\
\sum_{n \geq 0} S_{n, k}~ x^n &= x^k \times (\frac{A_1}{1-x} + \frac{A_2}{1-2x} + \dots +  \frac{A_k}{1-kx})
\end{align*}
Solving for $A_1, A_2, \dots A_k $ we'll get $A_r = (-1)^{k-r} \frac{r^{k-1}}{(r-1)! (k-r)!}$. \\
$S_{n,k}$ is the coefficient of $x^n$ in RHS. i.e., 
\begin{align*}
S_{n,k} &= coeff~ of ~x^n ~in~ x^k \times (\frac{A_1}{1-x} + \frac{A_2}{1-2x} + \dots +  \frac{A_k}{1-kx}) \\
&= coeff ~ of ~ x^{n-k} ~ in ~ \sum_{r=1}^k \frac{A_r}{1-rx}
\end{align*}
Coefficient of $x^{p}$ in $\frac{1}{1-rx} = r^p$. hence,
\begin{align*}
S_{n,k} &= \sum_{r=1}^k A_r r^{n-k} \\
&= \sum_{r=1}^k (-1)^{k-r} \frac{r^{k-1}}{(r-1)! (k-r)!} ~ r^{n-k}\\
S_{n,k} &= \sum_{r=1}^k (-1)^{k-r} \frac{r^{n}}{(r-1)! (k-r)!}
\end{align*}
The above expression is Stirling number of second kind
%\end{equation}
%$$ = S_{0,0}~x^0y^0 + {\sum_{n = 0, k \geq 1}^ {\infty} S_{0,k}~x^0y^k} + {\sum_{n \geq 1, k = 0}^ {\infty} S_{n,0}~x^ny^0} + {\sum_{n \geq 1, k \geq 1}^ {\infty} S_{n,k}~x^ny^k} $$ \\
%$&= 1 + \sum_{\substack{n \geq 1 \\ k \geq 1}} ^ {\infty} S_{n,k} ~ x^n y^k$

\section{Exponential generating functions}
In ordinary generating functions we associate sequence, ${(a_n)}_{n \geq 0}$ with $G(x) = \sum_{n \geq 0} a_n x^n$. In $G(x)$ we chose basis $\{ 1, x, x^2, x^3 \dots\}$ for set of all polynomials in one variable. But there are many other basis for set of polynomials, like $\{1, x, x(x-1), x(x-1)(x-2), \dots\}$. We chose basis $\{ 1, x, x^2, x^3 \dots\}$ because it has combinatorial meaning. Other such meaning full basis are $\{\frac{x^n}{n!}\}_{n \in \N}$, $\{e^{-x} \frac{x^n}{n!}\}_{n \in \N}$ and $\{\frac{1}{n^x}\}_{n \in \N}$. In this lecture we'll explore exponential generating functions which use basis $\{\frac{x^n}{n!}\}_{n \in \N}$ . \\
So  ${(a_n)}_{n \geq 0}$ is associated with $E(x) = \sum_{n \geq 0} a_n \frac{x^n}{n!}$ . Let's see a few examples - 
\begin{align*}
(1, 1, 1, \dots ) &\xrightarrow[generating function]{exponential} \sum_{n \geq 0} \frac{x^n}{n!} = e^x    \\
&\xrightarrow[generating function]{ordinary} \sum_{n \geq 0} x^n = \frac{1}{1-x} \\
(1!, 2!, 3!, \dots) &\xrightarrow[generating function]{exponential} \sum_{n \geq 0} n! \frac{x^n}{n!} = \sum_{n \geq 0} x^n = \frac{1}{1-x}
\end{align*}
$\frac{1}{1-x}$ is ordinary generating function(ogf) of $(1, 1, 1, \dots )$ and exponential generating function(egf) of $(1!, 2!, 3!, \dots )$.\\
\\
\textbf{\Large {Operations of EGF}}
\begin{parts}
\item{\textbf{Addition :} } It's similar to ogf.
    \begin{align*}
        \{a_n\}_{n \geq 0} &\xrightarrow[]{egf} E(x) \\
        \{b_n\}_{n \geq 0} &\xrightarrow[]{egf} F(x) \\
        \{a_n + b_n \}_{n \geq 0} &\xrightarrow[]{egf} E(x) + F(x) \\
    \end{align*}
\item {\textbf{Shifting :} } Multiplying ogf by $x$ shifts the sequence to left as seen in earlier lectures.  
    \begin{align*}
        \{a_0, a_1, a_2 \dots \} &\xrightarrow[]{egf} E(x) \\
        \{0, a_0, a_1, a_2 \dots \} &\xrightarrow[]{egf} xE(x) \\
    \end{align*}
Differentiating egf function will shift the sequence to right.
\begin{gather*}
    \{a_0, a_1, a_2 \dots \} \xrightarrow[]{egf} ~~ E(x) \endline
        \{a_1, a_2, a_3 \dots \} \xrightarrow[]{egf} \frac{d}{dx} E(x) \\
        \frac{d}{dx} E(x) = \sum_{n \geq 1} a_n~ \frac{n. x^{n-1}}{n!} = \sum_{n \geq 1} a_n~ \frac{x^{n-1}}{(n-1)!} =  \sum_{n \geq 0} a_{n+1}~ \frac{x^{n}}{n!}
\end{gather*}
\item {\textbf{Multiplication :} }   EGFs are used if the sequence counts labelled structures like permutations, derangements and partitions. Let $(a_n)_{n \geq 0}$, $(b_n)_{n \geq 0}$ count arrangements  of type $A$ and type $B$ respectively using $n$ labelled objects. If we want to count type $C$ arrangements, that can be obtained by a unique split of $n$ objects into two sets and then arranging first set according to type $A$ and second set according to type $B$ - \\
 No.of arrangements of type $C$ of size $n, c_n = \sum_{k=0}^n {n \choose k} a_k b_{n-k}$.\\
 Now Let's see how multiplication of $A(x)$(egf of $A$) and $B(x)$(egf of $B$) is useful 
 \begin{align*}
     A(x).B(x) &= (\sum_{n=0}^{\infty} a_n \frac{x^n}{n!})(\sum_{n=0}^{\infty} b_n \frac{x^n}{n!}) \\
     &= \sum_{n=0}^{\infty}(\sum_{k=0}^{n} \frac{a_k}{k!}.\frac{b_{n-k}}{(n-k)!}) x^n \\
     &= \sum_{n=0}^{\infty}(\sum_{k=0}^{n} \frac{n!}{k!(n-k)!} a_k b_{n-k}) \frac{x^n}{n!} \\
     &= \sum_{n=0}^{\infty}(\sum_{k=0}^{n} {n \choose k} a_k b_{n-k}) \frac{x^n}{n!} \\
     &= \sum_{n=0}^{\infty} c_n \frac{x^n}{n!} \\
     A(x).B(x) &= C(x)
 \end{align*}
 
 Now Let's see few examples of egf 
 \subsection{Derangements } Recall that we've discussed derangements in PIE and recurrence relations. Now let's derive it using egf. Let $D_n$ represent set of derangements of $n$ objects and let $d_n = |D_n|$. we can see that $d_0 = 1 ,~ d_1 = 0,~ d_2 = 1$. Recall the recurrence relation : $$ d_{n+2} = (n+1)(d_{n+1} + d_n)$$
 \begin{align*}
     D(x) &= \sum_{n=0}^{\infty} d_n \frac{x^n}{n!} \\
     D'(x) &= \sum_{n=0}^{\infty} d_{n+1} \frac{x^n}{n!} ~(by ~shifting ~operation)\\
     &= \sum_{n=1}^{\infty} n(d_n + d_{n-1}) \frac{x^n}{n!} \\
     &= \sum_{n=1}^{\infty} nd_n \frac{x^n}{n!} + \sum_{n=1}^{\infty} nd_{n-1} \frac{x^n}{n!} \\
     &= x \sum_{n=1}^{\infty} d_n \frac{x^{n-1}}{(n-1)!} + x \sum_{n=1}^{\infty} d_{n-1} \frac{x^{n-1}}{(n-1)!} \\
     &= x \sum_{n=0}^{\infty} d_{n+1} \frac{x^n}{n!} + x \sum_{n=0}^{\infty} d_n \frac{x^n}{n!} \\
     D'(x) &= xD'(x) + xD(x) \\
     (1-x)D'(x) &= xD(x) \\
     \frac{D'(x)}{D(x)} &= \frac{x}{1-x} = \frac{1}{1-x} - 1
 \end{align*}
Integrating on both sides
     $$\ln{D(x)} = \ln{(1-x)} - x + c$$
Since $D(0) = d_0 = 1 \implies c=0$.
    $$\ln{D(x)} = \ln{(1-x)} - x$$
    $$D(x) = \frac{e^{-x}}{1-x} = \sum_{n=0}^{\infty} d_n \frac{x^n}{n!}$$
To get $d_n$ we need coefficient of $\frac{x^n}{n!}$ in LHS.
$$e^{-x} \frac{1}{1-x} = (\sum_{n=0}^{\infty} (-1)^n \frac{x^n}{n!})(\sum_{n=0}^{\infty} n! \frac{x^n}{n!})$$
Coefficient of $\frac{x^n}{n!}$ by multiplication property = $\sum_{k=0}^{n} {n \choose k} a_k b_{n-k}$
\begin{align*}
    d_n = \sum_{k=0}^{n} {n \choose k} (-1)^{n-k} k!  
    = \sum_{k=0}^{n} (-1)^{n-k} k! {n \choose k}
\end{align*}
$d_n$ is count of derangements of $n$ objects.

\subsection{Bell Numbers}
Let $S_{n,k}$ represent number of ways of partitioning \{$1, 2 ,3 \dots n $\} into $k$ non empty blocks and $B_n$ represent number of ways of partitioning \{$1, 2 ,3 \dots n $\} ($B_0 = 1$). By definitions, 
$$B_n = \sum_{k=0}^n S_{n,k}$$
Equivalent intrepretation :  Consider a number whose prime factorization is square free i.e., $k \in \N$ such that $k = p_1p_2 \dots p_n$ where $\{p_1, p_2, \dots p_n\}$ are distinct primes. Number of ways of writing $k$ as product of natural numbers $\geq 2 = $ number of ways of partitioning  $\{p_1, p_2, \dots p_n\}$  $= B_n$. \\
Recurrence Relation : 
 $$ B_n = \sum_{k=0}^{n-1} {{n-1} \choose k} B_k $$ 
% LHS : Number ways of partitioning  \{$1, 2 ,3 \dots n $\}. \\
% RHS : \begin{parts}
% \item If $1$ occurs in singleton set. No.of such partitions  = $B_{n-1}$
% \item If $1$ occurs in set with two elements. No.of ways element can be chosen $ = n-1$.No.of such partitions  = ${{n-1} \choose 1}B_{n-2}$.
% \item $1$ occurs in set with $k+1$ elements. No.of ways elements in this set can be chosen $ = {n-1 \choose k}$. No.of such partitions  = ${{n-1} \choose k}B_{n-(k+1)}$.
 \end{parts}
% Hence total $= \sum_{k=0}^{n-1} {{n-1} \choose k} B_{n-k-1}  = \sum_{k=0}^{n-1} {{n-1} \choose n-k-1} B_{n-k-1} = \sum_{n-k-1=0}^{n-1} {{n-1} \choose k} B_{k} = \sum_{k=0}^{n-1} {{n-1} \choose k} B_{k} $ \\
 Let's derive closed form expression for $B_n$ :
 \begin{align*}
     B(x) &= \sum_{n=0}^{\infty} B_n \frac{x^n}{n!} \\
     B'(x) &= \sum_{n=0}^{\infty} B_{n+1} \frac{x^n}{n!} ~(by ~ shifting ~rule) \\
     &= \sum_{n=0}^{\infty} (\sum_{k=0}^{\n} {n \choose k} B_k) \frac{x^n}{n!} \\
     &= \sum_{n=0}^{\infty} (\sum_{k=0}^{\n} {n \choose k}. ~1 B_k) \frac{x^n}{n!} \\
     &= (\sum_{n=0}^{\infty} B_n \frac{x^n}{n!})(\sum_{n=0}^{\infty} 1 \frac{x^n}{n!}) ~(by ~ multiplication ~rule) \\
     B'(x) &= B(x) . e^x \\
     \frac{B'(x)}{B(x)} &= e^x
 \end{align*}
 Integrating on both sides 
 $$\ln{B(x)} = e^x + c$$
 Since $B(0) = B_0 = 1 \implies c = -1$. Hence,
 %$$ B(x) = e^{e^x - 1} = \frac{e^{e^x}}{e}$$
 \begin{align*}
     B(x) &= e^{e^x - 1} \\
     &= \frac{e^{e^x}}{e} \\
     &= \frac{1}{e} (\sum_{k=0}^{\infty} \frac{(e^x)^k}{k!}) \\
     &= \frac{1}{e} (\sum_{k=0}^{\infty} \frac{e^{kx}}{k!}) \\
     &= \frac{1}{e} (~\sum_{k=0}^{\infty} \frac{1}{k!}~ (\sum_{n=0}^{\infty} \frac{(kx)^n}{n!})~) \\
     &= \frac{1}{e} (~\sum_{n=0}^{\infty} \frac{x^n}{n!}~ (\sum_{k=0}^{\infty} \frac{k^n}{k!})~)\\
B(x)&= \sum_{n=0}^{\infty} \frac{1}{e} (\sum_{k=0}^{\infty} \frac{k^n}{k!}) \frac{x^n}{n!} \\
    B_n &= \frac{1}{e} (\sum_{k=0}^{\infty} \frac{k^n}{k!})
 \end{align*}
 We've derived closed form expression for bell number. Above expression for $B_n$ is also called as Dobinski's formula.
 
\end{parts}
