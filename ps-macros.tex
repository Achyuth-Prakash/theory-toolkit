\usepackage{hypcap}
\renewcommand{\E}{{\mathbb E}}

\usepackage{todonotes}
\newcounter{todocounter}
\newcommand{\todonum}[2][]{\stepcounter{todocounter}\todo[#1]{\thetodocounter: #2}}
\newcommand{\jsay}[1]{\todonum[inline,color=red!20]{\small Jayalal says: Todo - #1}}

% modified exercise enviornment to use with collect  package
\usepackage{enumerate}
\newcounter{excount}
\setcounter{excount}{0}
\theoremstyle{definition}
\newtheorem{ex}[section]{Exercise}
\newtheorem{curious}[theorem]{Curiosity}
% for exercises which are problem set questions

\newtheorem{exercise-prob}[section]{Exercise}
\theoremstyle{plain}	    
%%%%%%%%%%%%%%%%%%%% Exercise and pset macros (start) %%%%%%%%%%%%%%%%%%%%%%%5
% For problem set back reference.
\def\psetbackref{1}

% For Curiosity Drive
\definecollection{curious.tmp}
\makeatletter
\newenvironment{curiousity}
    {\@nameuse{collect*}{curious.tmp}
		  {\begin{curious}}
		    {\end{curious}}{}{}
    }{\@nameuse{endcollect*}}
\makeatother

% For exercise
\definecollection{ex.tmp}
\makeatletter
\newenvironment{exercise}
    {\@nameuse{collect*}{ex.tmp}
		  {\begin{ex}}
		    {\end{ex}}{}{}
    }{\@nameuse{endcollect*}}
\makeatother


%%%%%%%%%%%%%%%%%%%%%%%%%%%%%%%%%%%%%%%%%%%%%%%%%%%%%%%%%%%%%%
% To create a a new pset with pset number n, copy paste the following code
% with XX replaced by n.
%
% 	 \definecollection{psXX.tmp}
%	 \makeatletter
%	 \newenvironment{show-psXX}[1]
%	     {\@nameuse{collect*}{psXX.tmp}
%	         {\ifthenelse{ \equal{\psetbackref}{1} }{\label{prob:#1}}{}} {}
%   	         {\item \label{#1} (See Exercise~\ref{prob:#1})} {}
%	     }{\@nameuse{endcollect*}}
%	 \makeatother
%
%	 \makeatletter
%	 \newenvironment{psXX}
%	     {\@nameuse{collect}{psXX.tmp}
%			{\item}{}
%	     }{\@nameuse{endcollect}}
%	 \makeatother
%
% 

%%% For ps1 
\definecollection{ps1.tmp}
\makeatletter
\newenvironment{show-ps1}[1]
    {\@nameuse{collect*}{ps1.tmp}
	    {\ifthenelse{ \equal{\psetbackref}{1} }			{\label{prob:#1}}{}} {}
	    {\item \label{#1} (See Exercise~\ref{prob:#1})} {}
    }{\@nameuse{endcollect*}}
\makeatother

\makeatletter
\newenvironment{ps1}
    {\@nameuse{collect}{ps1.tmp}
		{\item }{}
    }{\@nameuse{endcollect}}
\makeatother

%%% For ps2
\definecollection{ps2.tmp}
\makeatletter
\newenvironment{show-ps2}[1]
    {\@nameuse{collect*}{ps2.tmp}
	    {\ifthenelse{ \equal{\psetbackref}{1} }{\label{prob:#1}}{}} {}
	    {\item \label{#1} (See Exercise~\ref{prob:#1})} {}
    }{\@nameuse{endcollect*}}
\makeatother

\makeatletter
\newenvironment{ps2}
    {\@nameuse{collect}{ps2.tmp}
		{\item}{}
    }{\@nameuse{endcollect}}
\makeatother

%%% For ps3
\definecollection{ps3.tmp}
\makeatletter
\newenvironment{show-ps3}[1]
    {\@nameuse{collect*}{ps3.tmp}
	    {\ifthenelse{ \equal{\psetbackref}{1} }{\label{prob:#1}}{}} {}
	    {\item \label{#1} (See Exercise~\ref{prob:#1})} {}
    }{\@nameuse{endcollect*}}
\makeatother

\makeatletter
\newenvironment{ps3}
    {\@nameuse{collect}{ps3.tmp}
		{\item}{}
    }{\@nameuse{endcollect}}
\makeatother


%%% For ps3
\definecollection{ps4.tmp}
\makeatletter
\newenvironment{show-ps4}[1]
    {\@nameuse{collect*}{ps4.tmp}
	    {\ifthenelse{ \equal{\psetbackref}{1} }{\label{prob:#1}}{}} {}
	    {\item \label{#1} (See Exercise~\ref{prob:#1})} {}
    }{\@nameuse{endcollect*}}
\makeatother

\makeatletter
\newenvironment{ps4}
    {\@nameuse{collect}{ps4.tmp}
		{\item}{}
    }{\@nameuse{endcollect}}
\makeatother

%%%%% Exercise and pset macros (end) %%%%%
